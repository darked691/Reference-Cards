% Clément CAUMES 21501810
% Yassin Doudouh 21500127
% 05 mai 2017
% Projet Metro IN403

% Fichier de rapport du projet METRO

\documentclass[a4]{article}

\usepackage[utf8]{inputenc}
\usepackage[french]{babel}
\usepackage{amssymb}
\usepackage{caption}
\usepackage{amsmath}

\author{Clément Caumes 21501810 - Yassin Doudouh 21500127}
\title{Rapport du projet Metro}
\date{05 mai 2017}

\begin{document}
\maketitle

\section{Description des structures de données}

Les déclarations de ces structures de données sont dans \textit{graphe.h}. \newline

La structure \textbf{SOMMET} contient les informations d'un sommet. Il y a le nom de la station, le numero du sommet dans metro.txt, le numero
de la ligne dans le réseau du metro parisien, la géolocalisation sur la fenêtre graphique de l'application.\newline

La structure \textbf{ARC} contient les informations d'un arc entre deux sommets, avec le numéro du sommet du terminus et la pondération de l'arc (temps qu'il faut pour aller d'une station à une autre).\newline

La structure \textbf{DIJKSTRA}, qui stockera le plus court chemin calculé avec l'algorithme de Dijkstra, contient le numéro du sommet de départ, le numéro du sommet d'arrivée. 
Il y a également la durée qui sera calculée dans cree chemin et la liste des stations qu'il faut traverser pour aller du départ à la destination voulu.\newline

La structure \textbf{GRAPHE} contient toutes les informations du graphe de l'application : un tableau de 2 sommets avec les 2 sommets choisis par l'utilisateur pour le calcul du plus court chemin, un entier représentant 
le nombre de sommets dans le graphe, un tableau des sommets avec leur nom et leur numéro de ligne, et un tableau à 2 dimensions où chaque arc est déterminés avec 
pondération.\newline

La structure \textbf{METRO} contient toutes les informations relatives à l'application : le graphe qui représente le metro parisien, l'affichage (si on zoom ou non) et 
le mode (si on a quitté l'application).

\section{Préparation du projet}
L'application que nous devions créer devait permettre de choisir des stations de metro au clic afin de calculer le plus court chemin entre ces stations.\newline
Nous avons rajouté quelques données dans le fichier metro.txt. En effet, il était intéressant de connaître les lignes de chaque station ainsi que leur position sur la carte. 
Les terminus pour chaque arc étaient également une donnée intéressante pour connaître les directions. 

\section{Fonctionnement général}
Tout d'abord, on initialise le graphe du metro. On alloue le bon nombre de sommets grâce 
à la lecture dans le fichier metro.txt. Les arcs sont nuls pour le moment. 
Cela correspond à la procédure \emph{initialise\_metro} dans \textit{graphe.c}.\newline

Le tableau des sommets est alors initialisé pour contenir les informations de chaque station 
(le nom de la station, le numéro du sommet, le numéro de sa ligne et ses coordonnées sur l'interface graphique).
Cette procédure est faite par \emph{initialise\_stations} avec la lecture de metro.txt dans \textit{lire\_ecrire.c}. \newline

Le tableau des arcs entre les sommets est aussi initialisé avec le numéro du terminus (direction de l'arc sur la ligne) 
et la pondération de cet arc. Sachant que le metro est à double sens, l'arc opposé est également initialisé avec la même pondération 
mais avec le terminus opposé. \emph{initialise\_reseau} fait ce travail grâce à metro.txt dans \textit{lire\_ecrire.c}.\newline

En fonction de ce que veut faire l'utilisateur, il y a plusieurs manipulations (avec \emph{interaction} de \textit{affichage.c}) : 
\begin{itemize}
\item cliquer sur la touche '\textbf{Q}' pour quitter.
\item cliquer sur la touche '\textbf{Z}' pour zoomer. Ici, les flèches directionnelles sont utilisables pour se déplacer. 
Recliquer sur '\textbf{Z}' permet de dézoomer. 
\item le clic sur les différents sommets permet de sélectionner/désélectionner 
les stations que l'on a choisi pour calculer le plus court chemin (avec \emph{ajoute\_supprime\_selection\_sommet} de \textit{affichage.c}). Ce chemin est calculé 
quand l'utilisateur a choisi 2 sommets et l'application utilise l'algorithme de Dijkstra avec \emph{calcul\_dijkstra} de \textit{graphe.c}. 
Quand le tableau des successeurs de Dijkstra a été initialisé, on crée une liste de sommets par lesquels il 
faut passer pour avoir le plus court chemin. \newline
\end{itemize}

Le plus court chemin est maintenant trouvé. \emph{ecrit\_chemin} de \textit{lire\_ecrire.c} écrit dans le terminal les stations par lesquelles il faut passer et le temps qu'il faut. 
Les stations sur la carte sont mises en valeur en jaune avec \emph{affiche\_chemin} dans affichage.c.

Finalement, la mémoire est libérée quand on quitte l'application avec \emph{libere\_graphe}, 
\emph{libere\_chemin\_dijkstra} et \emph{libere\_fenetre}.\newline

\section{Principale procédure : Algorithme Dijkstra}
A première vue, l'algorithme qui était le plus adapté pour calculer le plus court chemin 
entre des stations de metro était celui de Floyd-Warshall car il calcule entre toutes les paires de sommets. 
Pourtant, sa complexité est de o($n^{3}$) avec n le nombre de sommets. \newline
Mais, l'algorithme de Dijkstra a une complexité de o($n^{2}$) : à chaque itération on traite un sommet (o($n$)) et à chaque traitement de sommet, on parcourt 
le tableau des plus courtes distances (o($n$)). 
Et s'il fallait atteindre la même que Floyd-Warshall, il aurait fallu n fois chemins différents 
o($n^{2}\times n$) : ce qui n'est pas réaliste pour 375 stations. \newline
C'est pour cette raison que nous avons choisi Dijkstra pour calculer le plus court chemin entre stations. \newline \newline \newline

Son implémentation est la suivante : 
\begin{itemize}
\item d'abord, on stocke le fait que aucun sommet n'ait été traité, aucun 
sommet n'ait encore de père, les plus petites distances sont encore infinis. 
\item on traite le sommet de départ et on met à jour les plus petites distances 
avec ses successeurs en utilisant les pondérations. 
\item tant qu'on a des sommets à traiter, on prend la plus petite des plus petites distances 
et on regarde si on améliore les plus petites distances de ces successeurs en passant par ce sommet. 
Si c'est le cas alors ce dernier devient son père. 
\item le tableau des pères est alors initialisé complétement.
\end{itemize}

\section{Améliorations}
Lorsque l'on est dans le mode zoom, on ne peut pas sélectionner de stations car les coordonnées ont changées. 
En effet, il aurait été difficile de recalculer les coordonnées de chaque station puisque la division de la carte 
n'est pas parfaitement symétrique. On aurait pu rencontrer des problèmes de sélections (décalages de coordonnées).

\end{document}
